\subsection{Policy Report Guidelines}\label{policy-report-guidelines}

Generally, a policy report provides analysis and recommendations for a
problem or policy issue. Your policy report should provide a concise
summary of the policy problem, current policies, the major stakeholders,
some policy options to address it, and a recommendation for the best
option. Policy reports are generally concise documents; therefore, I will expect
well organized, clear, concise, and professional quality writing.
When working on your policy report, you should consider yourself an \emph{objective technician} 
that provides expertise driven advise to a decision-maker, rather than as an advocate.

\hypertarget{paper-guidelines}{%
\subsection{Policy Report Draft Sections}\label{paper-guidelines}}

Each part of the policy report will focus on a different aspect of the one problem or
policy issue you choose. You should include at least two \emph{academic}
sources (not including course readings) for each draft section. Appropriate
sources include (for example, but not limited to), peer-reviewed journal
articles, books, government reports, and/or research institute (``think
tank'') reports.

\hypertarget{issue-paper-1-define-the-problem}{%
\subsubsection{Part 1: Define the
Problem}\label{part-1-define-the-problem}}

In part 1, you should provide a brief overview of your policy
issue to introduce the reader to the issue. It can include answers to
such questions as, why is it an important issue? What makes it a
societal problem? Why should policy makers be concerned about the issue?
Define some of the key terms with regard to the issue. Make clear if you
are examining an issue on the local, state, or federal level.

\begin{itemize}
\item \textbf{Use the Bardach and Patashnik book, pgs 1-14 (6th edition) -- \textit{Step One: Define the Problem} -- for guidance.}

\item \textbf{Draft due: Sept 20}
\end{itemize}

\hypertarget{issue-paper-2-policy-history}{%
\subsubsection{Part 2: Policy
History and Policy Mapping}\label{issue-paper-2-policy-history}}

In part 2, you should discuss what policy or policies are
currently in place to address your issue. What institutions and
subsystems are active in policymaking with regard to your issue. Is
there legislation? Who implements it? Are there regulations?

\begin{itemize}
\item \textbf{Use the Pennock reading and policy mapping reading provided on OAKS as guidance. Note, no
need to address it to anyone.}

\item \textbf{Draft due: Sept 27}
\end{itemize}

\hypertarget{issue-paper-4-policy-alternatives}{%
\subsubsection{Part 3: Policy
Alternatives}\label{issue-paper-4-policy-alternatives}}

In part 3, you should briefly describe three policy alternatives
that could be implemented to address your policy issue. In your
description, be sure and describe the causal model for each approach as
well as the pros and cons of each approach.

\begin{itemize}
\item \textbf{Use the Bardach and Patashnik book, pgs 21-31 (\textit{Step Three: Construct the Alternatives}) as guidance.}

\item \textbf{Draft due: Nov 23}
\end{itemize}

\hypertarget{issue-paper-5-consideration-of-alternatives}{%
\subsubsection{Paper 4: Consideration of
Alternatives and Recommendation}\label{issue-paper-5-consideration-of-alternatives}}

In part 4, you should evaluate each of the three policy
alternatives from paper 4 according to the criteria of efficiency,
political feasibility, and administrative capacity to implement. Create a table with the criteria as columns, each policy alternative as the rows, and a short sentence or two in the cells. Your table should be set-up like pg 156 of the Radin reading. Then, project (make guesses about) the expected impact of each alternative on your policy problem. Finally, make a recommendation as to which of the three policy alternatives should be chosen. Justify your choice based on your evaluation of the alternatives according to the three criteria. 

\begin{itemize}
\item \textbf{Use the Radin reading as well as Bardach and Patashnik book, pgs 31-77 (\textit{Steps Four, Five, and Six}) as guidance.}

\item \textbf{Draft due: Nov 30}
\end{itemize}

\subsection{Final Policy Report}

For the final policy report, you should have at least 10 academic sources. These include (for example,
but not limited to), peer-reviewed journal articles, books, government
reports, and/or research institute (``think tank'') reports. Readings
from class may be used but only in addition to the 10 required. The 10
can include references used for your draft sections.

\vspace{0.10in}
\noindent Use in-text citations OR footnotes to cite your sources and provide a
bibliography page at the end. Note, the bibliography page does not count
as page part of the length requirement. The length requirement is not
arbitrary, it is what is expected if all the required information is
presented in a concise manner. If you are under 10 or over 15, you have
more work to do. I am happy to assist you in editing the report. Also,
be sure you have incorporated any feedback I provided about the relevant
drafts in your final report.

\hypertarget{outline}{%
\subsection{Outline}\label{outline}}

Your policy report should follow the outline below and contain each of
the following elements. \textbf{Use the following headings in your
report}.

\hypertarget{introduction-and-problem-definition}{%
\subsubsection{Introduction and Problem
Definition}\label{introduction-and-problem-definition}}

The introduction should provide an overview of your policy issue that
defines the policy problem, provides important contextual background
information, and explains what the report will cover. Why is it an
important issue? Why is it an interesting issue? What is the situation
that warrants definition as a public problem?

\begin{itemize}
\item About 2 pages
\end{itemize}

\hypertarget{issue-analysis}{%
\subsubsection{Policy History and Mapping}\label{issue-analysis}}

This section should provide some historical context for the policy
issue, highlight how the problem developed, and discuss the major
stakeholders involved. What are the primary controversies? What are the
major federal, state, and/or local policies (i.e., legislation,
regulations, or Supreme Court cases) that are associated with your topic
area? Who are the primary actors and/or stakeholders involved in your
issue? What are their positions and how do they define the problem? What
does the scientific literature say about your topic? Are there any
scientific controversies?

\begin{itemize}
\item
  About 3 pages
\end{itemize}

\hypertarget{policy-alternatives}{%
\subsubsection{Policy Alternatives}\label{policy-alternatives}}

Consider three policy alternatives to address the problem and add doing
nothing (status-quo) as an option. In your description of each
alternative, describe the causal model for each approach as well as the
pros and cons of each approach. Next, evaluate each of the three policy
alternatives according to the criteria of efficiency, political
feasibility, and administrative capacity to implement. Create a table with the criteria as columns, each policy alternative as the rows, and a short sentence or two in the cells. Then, project
(make guesses about) the expected impact of each alternative on your
policy problem.

\begin{itemize}
\item
  Note: combine your drafts of parts 3 and 4 
\item
  About 3 to 4 pages
\end{itemize}

\hypertarget{recommendation-and-implementation}{%
\subsubsection{Recommendation}\label{recommendation-and-implementation}}

Based on your analysis of the various policy alternatives, make a
recommendation for a course of action. Why should the decision-maker
choose that option over others? 

\begin{itemize}

\item
  About two pages
\end{itemize}

\hypertarget{bibliography}{%
\subsubsection{Bibliography}\label{bibliography}}

List the sources you cited in alphabetical order. You must have at least
10 beyond the class readings. Use any citation style you prefer.



