\hypertarget{covid-19}{%
\section{COVID-19}\label{covid-19}}

The COVID-19 pandemic is still ongoing. \textbf{The College of
Charleston requires that masks be worn while indoors and you must wear a
mask at all times while in class.} Although vaccinations are currently
not required, \emph{I ask you to be respectful of the health and safety
of others}. If you have not received the \textbf{COVID-19 vaccine, which
is safe, free, and effective, please consider doing so immediately}.
Information about the vaccine is available from the
\href{https://scdhec.gov/covid19/covid-19-vaccine}{SCDHEC website} and
information about where and when to obtain a vaccine is also available
on the SCDHEC website \href{https://vaxlocator.dhec.sc.gov/}{vaccine
locator page}.

\hypertarget{course-description}{%
\section{Course Description}\label{course-description}}

\begin{quote}
\emph{Our responsibility is one of decision---for to govern is to
choose} - John F. Kennedy
\end{quote}

\vspace{0.1in}

\noindent The above quote from President Kennedy came from a speech at
Liberty Hall in Philadelphia on July 4th, 1962. In the audience were
members of the 54th National Governors' Conference. Speaking to elected
officials that were charged with making often difficult decisions,
Kennedy was reminding them that making choices is at the heart of
governing. How do policymakers make choices regarding public policy?
This course will address this question by examining the policymaking
process in the United States.

\vspace{0.1in}

\noindent Public policy is an on-going process of actions (or inactions)
and decisions (or non-decisions) by government at all levels. The
policymaking process considers the interactions of decision-makers,
those trying to influence decision-makers, and the venues where
decisions are made. As likely future actors in the complex process of
making public policy, you will likely be involved in considering,
informing, and perhaps making consequential policy decisions. The aim of
this course is for students in the MES, MPA, and concurrent MES/MPA
programs to develop an understanding of the ways in which policy choices
are developed, considered, and made in a democratic system.

\vspace{0.1in}

\noindent The tools and frameworks discussed in the course will help you
to be able to identify the major components and important facets of the
policymaking process as well as enhance your capacity to critically
analyze important policy issues.

\vspace{0.1in}

\noindent The course is divided into three sections, 1)
\textbf{Foundations}: includes some of the foundational aspects of
policymaking in the US such as democracy, equality, institutions, and
actors; 2) \textbf{Theories of the Policy Process}: covers the major
policy process theories that scholars have developed; and 3)
\textbf{Applied Policy Research}: discusses how knowledge is applied to
address pressing societal problems and includes policy design,
implementation, analysis, and evaluation.

\vspace{0.1in}

\noindent Laptops are allowed, but should only be used to access the
readings. Phones should be put away during class. \emph{I encourage you
to take notes by hand, with pen and paper}.
\href{https://www.nytimes.com/2017/11/27/learning/should-teachers-and-professors-ban-student-use-of-laptops-in-class.html}{You
learn better that way}. I recommend taking notes using the
\href{http://www.usu.edu/arc/idea_sheets/pdf/note_taking_cornell.pdf}{Cornell
Method}.

\hypertarget{naspaa-competencies-and-course-learning-objectives}{%
\subsection{NASPAA Competencies and Course Learning
Objectives}\label{naspaa-competencies-and-course-learning-objectives}}

Students who graduate from NASPAA accredited MPA programs should develop
the ability to: lead and manage in public governance, participate in and
contribute to the policy process, apply skills in analysis and critical
thinking to solve problems and make decisions, articulate and apply a
public service perspective, and communicate and interact productively
with a diverse and changing workforce and citizenry. \textbf{This course
is designed with a special emphasis on providing students with the
knowledge, skills, and attitudes needed to participate in and contribute
to the policy process}. To that end, learning objectives include:

\begin{itemize}
\item
  Explain the various policy process theories and frameworks
\item
  Apply one or more of the policy process theories to a policy problem
\item
  Analyze policy problems and their potential solutions
\item
  Develop a detailed understanding of a policy problem or issue
\item
  Display oral, written, and group communication skills
\end{itemize}

\vspace{0.10in}

\noindent These objectives will be achieved through critically reading
the course readings; by writing several short papers; and by completing
a policy report about a particular policy issue.

\hypertarget{required-texts}{%
\section{Required Texts}\label{required-texts}}

The following texts are \textbf{required}. Additional readings will be
listed on the schedule below and available on OAKS.

\begin{itemize}
\item
  Birkland, Thomas A. 2020. \emph{An Introduction to the Policy Process:
  Theories, Concepts, and Models of Public Policy Making}. Routledge.
  5th Edition. (The 4th Edition from 2016 is also acceptable)
\item
  Weible, Christopher and Paul Sabatier (eds). 2018. \emph{Theories of
  the Policy Process}. Westview Press. 4th Edition.
\item
  Bardach, Eugene and Eric M. Patashnik. 2020. \emph{A Practical Guide
  for Policy Analysis: The Eightfold Path to More Effective Problem
  Solving}. SAGE Press. 6th Edition. (Older editions are also
  acceptable)
\end{itemize}

\hypertarget{attendance-policy}{%
\subsection{Attendance Policy}\label{attendance-policy}}

Attendance will be taken for each class session, and will be part of
your course engagement grade. You are allowed to miss \emph{one class
without penalty}. \textbf{However, do not come to class if you feel
ill.~Additionally, if you have been exposed to or tested positive for
COVID-19, do not come to class regardless of how you feel. In those
cases, I am happy to meet with you on Zoom to discuss material you
missed and wave the attendance requirement. Just let me know.}

\hypertarget{course-requirements-and-grading}{%
\section{Course Requirements and
Grading}\label{course-requirements-and-grading}}

Performance in this course will be evaluated on the basis of 10
reflection papers, draft sections of a policy report, a final policy
report, a decision memo, a course reflection paper, and class
engagement. \emph{Instructions for each assignment will be placed on
OAKS}. Due dates are in the schedule below. Points will be distributed
as follows:

\vspace{0.10in}
\begin{tabular}{ l l}
\hline
Assignment & Possible Points \\
\hline 
Reflection Papers (10 at 20 points each) & 200 points total \\
Policy Report Draft Sections (4 at 50 points each) & 200 points total \\ 
Policy Report & 100 points \\
Decision Memo & 75 points \\
Course Reflection Essay & 25 points \\
Engagement & 100 points \\ 
\hline
Total & 700 points \\
\hline
\end{tabular}

\vspace{0.10in}

\noindent \emph{There are 700 possible points for this course. Grades
will be allocated based on your earned points and calculated as a
percentage of 700}: A = 90 to 100\%; B+ = 87 to 89\%; B = 80 to 86\%; C+
= 77 to 79\%; C = 70 to 76\%; F = 69\% and below

\hypertarget{assignments}{%
\subsection{Assignments}\label{assignments}}

\textbf{Specific instructions for the following assignments are posted
on \href{https://lms.cofc.edu/d2l/home}{OAKS}. All work must be turned
in through the Assignment folder on OAKS, and is due at class time:
Monday, 5:30 PM Eastern}.

\begin{itemize}
\item
  \emph{Reflection Papers}: You will write 10 short, about 2 pages,
  reflection papers that summarize and integrate the readings. Prompts
  will be given for each paper in \href{https://lms.cofc.edu}{OAKS}.
  Note that 12 are assigned, but only 10 will be graded. \textbf{When
  assigned, reflection papers are due at class time.}

  \vspace{0.10in} \noindent We will learn about several of the leading
  theories of the policy process. To help facilitate learning, the
  reflection papers for the policy process theories will include briefly
  explaining some of the major components of each of the theories and
  discussing what you learned about the policy process from learning
  about the theory. You will also need to find a find a peer-reviewed
  journal article that applies the theory and post a brief (about a
  paragraph) summary of the article as well as a pdf of the article on
  the \href{https://lms.cofc.edu}{OAKS} discussion board.
\end{itemize}

\begin{itemize}
\item
  \emph{Policy Report}: You will write a 10-12 page policy report. For
  the policy report you will define a problem; explore the policy
  history and policy mapping of the problem; consider three alternatives
  to address the problem; evaluate those alternatives; and make a
  recommendation.

  \vspace{0.10in} \noindent You will turn in drafts of parts of the
  policy report throughout the semester. Instructions for the report and
  each part of the report is on \href{https://lms.cofc.edu}{OAKS}.

  \begin{itemize}
  
  \item
    \textbf{Part 1 draft DUE: Sept 20}
  \item
    \textbf{Part 2 draft DUE: Sept 27}
  \item
    \textbf{Part 3 draft DUE: Nov 22}
  \item
    \textbf{Part 4 draft DUE: Nov 29}
  \item
    \textbf{Final policy report DUE: Dec 10}
  \end{itemize}
\item
  \emph{Decision Memo}: The decision memo is based on your policy
  report, and will be a two-page memo where you clearly distill for a
  decision-maker what you have learned and would recommend regarding
  your policy issue. \textbf{The decision memo is due December 10.}
\item
  \emph{Course Reflection Paper}: The course reflection paper will be a
  1 to 2 page paper that will ask you guided questions so you can
  reflect on how well we met the course objectives. \textbf{The course
  reflection paper is due December 6.}
\item
  \emph{Course Engagement}: Students are expected to participate in the
  course by asking questions, providing thoughtful comments, and through
  making contributions to the group discussion portion of class.
  \textbf{Class discussion should be better than it would have been had
  you not attended.} Note that the professor has final say over what
  does or does not count as adequate participation.
\end{itemize}

\hypertarget{a-note-on-feedback}{%
\subsubsection{A Note on Feedback}\label{a-note-on-feedback}}

Each of your reflection papers will be graded in a holistic fashion. I
will examine and grade the document as whole, I am therefore not likely
to make specific comments on multiple aspects of each of your papers.
Any feedback will be provided on \href{https://lms.cofc.edu}{OAKS}. For
the reflection papers that discuss the theories of the policy process, I
will be looking to see that you adequately address each of the
questions, and if you do you will likely get all or most of the points.
For the rough drafts of sections of the policy report, I will provide
feedback on how to improve the draft for the final report. If you
adequately address what is asked for each of the papers, you will likely
get all of most of the points. Any major issues with the assignments or
your writing in general (such as grammar) that results in a significant
loss of points (below 90\%) will be addressed with feedback. Finally, I
am happy to meet you with to discuss any individual paper or your
writing as a whole.

\hypertarget{late-work-policy}{%
\subsection{Late Work Policy}\label{late-work-policy}}

Late work is subject to a 48-hour grace period, and after that will be
penalized 10\% each day (24 hr period) it is late, up to 3 days. After 3
days the assignment will not be accepted. For example, if an assignment
is due Monday at 5:30 PM, the grace period ends on Wednesday at 5:30 PM
and it is late as of 5:31 PM and you lose 10\%. After Thursday at 5:31
PM you lose another 10\%, after Friday at 5:31 PM another 10\%, and no
work will be accepted after Saturday at 5:30 PM. \emph{No late work will
accepted 72 hrs after the assignment due date and time}.
